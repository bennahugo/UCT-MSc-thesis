\usepackage[utf8]{inputenc}
\usepackage{fullpage}
\usepackage[toc,page]{appendix}
\usepackage{amsmath}
\usepackage{amsfonts}
\usepackage{mathrsfs}
\usepackage{tikz}
\usepackage{graphicx}
\usepackage{mdframed}
\usepackage{varwidth}
\usepackage{color}
\usepackage{listing}
\usepackage{wrapfig}
\usepackage[parfill]{parskip}
\usepackage{subcaption}
\usepackage{hyperref}
\usepackage[font={small}]{caption}
\usepackage{marvosym}
\usepackage{algorithmic}
\usepackage{algorithm}
\usepackage{pgfplots, pgfplotstable}
%-----------------------------------------------------
% Drawing packages
%-----------------------------------------------------
\usepackage[usenames,dvipsnames]{pstricks}
\usetikzlibrary{calc}
\usepackage{epsfig}
\usepackage{auto-pst-pdf}
\usepackage{pst-grad} % For gradients
\usepackage{pst-plot} % For axes
%-----------------------------------------------------
%User-defined operators
%-----------------------------------------------------
\DeclareMathOperator*{\sinc}{sinc}
\DeclareMathOperator*{\rms}{rms}
\newcommand{\psf}{\textrm{PSF}}
\newcommand{\III}{\textrm{III}}
\newcommand{\fracof}{\textrm{frac}}
\newcommand{\round}{\textrm{round}}
%-----------------------------------------------------
%Custom Tikz commands
%-----------------------------------------------------
\usetikzlibrary{shapes.geometric, arrows, positioning}
\tikzset{whitecircle/.style={draw=black, circle, fill=white, minimum width=0.5cm, maximum width=0.5cm},
blackcircle/.style={draw=black, circle, fill=black, minimum width=0.3cm, maximum width=0.3},
circcircle/.style={
	whitecircle,
	append after command={
	    \pgfextra{\let\mylastnode\tikzlastnode} 
             node [blackcircle, above = -0.175cm of \mylastnode.center] {}
        }
    }
}
\tikzstyle{start} = [blackcircle]
\tikzstyle{stop} = [circcircle]
\tikzstyle{process} = [rectangle, minimum width=3cm, minimum height=1cm, text width=3cm, text centered, draw=black, fill=orange!30]
\tikzstyle{decision} = [diamond, minimum width=3cm, minimum height=1cm, text centered, draw=black, fill=green!30]
\tikzstyle{rarrow} = [thick,->,>=stealth]