\chapter{Introduction}
\section{The synthesis imaging widefield problem}
In radio astronomy there is a well-known Fourier relationship between the sky brightness distribution and the measurements taken by antenna arrays, see for instance 
Clark \cite[Lecture 1]{taylor1999synthesis}. To synthesize an image of the sky intensity distribution the measurements taken by these so-called radio \textit{interferometers}
are normalle inverted by resampling the points onto a regularly-spaced grid and performing an inverse Fast Fourier Transform \cite{cochran1967fast} on the resampled measurements see for instance 
Thompson \cite{thompson1974interpolation}. Due to sparsity of the sampling domain the images are convolved with the inverse of the sampling pattern of the array, requiring that
the inversion step be called upon multiple times in an iterative deconvolution strategy such as Cotton-Schwab CLEAN (see Thompson et al. \cite[ch 11]{thompson2008interferometry}).

When synthesizing wide field images using an array of telescopes with non-East-West antenna pairs (``baselines'') the simple Fourier relationship between the sky brightness 
distribution and interferometer measurements break down since the Fast Fourier Transform only approximates the sky by a plane and the measurements taken by the telescope do not
remain coplanar over the course of longer observations. Due to this ``tilting'' of the sampling plane the projected position of sources far away from the centre of the synthesized 
fields do not remain constant and the brightness of the sources is smeared out as more measurements are integrated into the image. These two sources of error is collectively known
as the problem of non-coplanar widefield imaging, see for instance Cornwell and Perley \cite{cornwell1992radio}.

One strategy for resolving the sources affected by this smearing is to split the sky up into small narrow field (``facet'') images, tiling the sky in a 
polyhedron-like fashion \cite{cornwell1992radio}, since the degree to which sources are smeared depends on the distance those sources are from the individual facet image centres. The synthesized
images produced by such a non-coplanar faceting approach is hard to deconvolve and require reprojection and intensity correction in the overlapping areas. Another strategy uses 
convolution to introduce a correcting phase shift into each of the individual measurements taken by the instrument over time and is known as w-projection \cite{cornwell2008noncoplanar}. This approach
relates the non-coplanar measurements to a single plane, eliminating the phase delay introduced by the tilted interferometer baselines. 

In w-projection the resampling cost can be related to the size of the produced images, where as in traditional faceting the cost rise with number of facets. Unlike w-projection faceting 
is less memory intensive, especially for large images and arrays with very long baseline components. The work in this thesis concentrates on creating coplanar facet images by combining facet 
imaging and w-projection in what we call \textit{w-faceting}. Due to the nature of imaging using interferometers it is desirable to have baselines as long as possible to improve the resolving
capability of the telescope, enough baselines in-between to create uniform sampling coverage in the measurement domain and as much collecting area as possible to improve the sensitivity of 
the instrument. Instruments such as the Square Kilometre Array \cite{carilli2004science} and its pathfinders MeerKAT \cite{booth2009meerkat} and ASKAP \cite{johnston2008science} have significantly more
baselines than their predecessors such as the Jansky Very Large Array \cite{2041-8205-739-1-L1}. As the baselines grow roughly as the square of the number of antennas, and improved spatial resolving 
capability and sensitivity decreases the integration time for each measurement made per baseline the data rates from these telescopes provide a tough computational problem for image synthesis. Luckily due to
the linearity of the underlying relationship between the sky and the measurements, and dimensions of the input data and output products, the synthesis pipeline lends itself to the parallel and 
distributed nature of modern processing equipment. The work in this thesis focus on investigating resampling scalability within shared memory environments, comparing CPU- and GPU-based w-facet imaging.
Distributed implementation is not within the scope of this work, although the shared memory w-facet technique can be expanded to include multiple processing cluster nodes due to its data-parallel 
nature.
\section{Research questions and aims}
We have identified the following research questions for our research:
\begin{enumerate}
 \item Is GPU-based w-faceting more scalable than a CPU-based parallel and vectorized implementation?
 \item Does single precision gridding introduce any significant error in the synthesized images?
 \item What effect does double precision resampling have on resampling performance?
\end{enumerate}


\section{Approach}
\section{Outline}
