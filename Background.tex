\documentclass[a4paper,10pt]{report}
\usepackage[utf8]{inputenc}
\usepackage{fullpage}
\usepackage{amsmath}
\usepackage{amsfonts}
\usepackage{tikz}
\usepackage{graphicx}
\usepackage{mdframed}
\usepackage{color}
\usepackage{listing}
% Title Page
\title{Facet-based Imaging}
\author{Benjamin Hugo}
\date{April 2014}

\begin{document}
\maketitle

\section{Introduction}
\subsection{Radio astronomy}
Through eons of visual observation mankind learned the laws that govern the movement of the celestrial objects in the skies above him. In the last 500 years we've 
seen major advances in our ability to observe new phenomina, starting with the first optical telescope and manufacture of transparent glass. However these observations
were still restricted to the wavelengths of visible light. The next series of improvements came only by the 1930s when observations at longer wavelengths were made. This was
after several failed attempts to receive radio emissions from the sun by the late nineteenth century. Herschel (1930) expanded the observable electromagnetic spectrum to the 
near infrared wavelengths: $0.35\mu m \leq\lambda\leq 1\mu m$. The most drastic change came when Jansky (1931) observed radiation from an extraterrestrial source (which 
wasn't the sun) with much longer wavelengths of $14.6m$ using a direction sensitive antenna array. By 1937 Grote Reber published the first observations in a professional astronomical
journal. After World War II improved equipment and the use of the optical interferometer ultimately lead to the techniques of aperture synthesis still employed today. Ultimately all
historical development has been geared toward shorter wavelengths, higher sensitivity, and higher angular resolution\cite{christiansenradiotelescopes,wilson2009tools}.

Even since those early days of optical telescopes it became apparent that the aperatures employed in telescopes have to be much larger than the wavelength being observed by
these telescopes.
\bibliography{bibfile}
\bibliographystyle{plain}
\end{document}          
